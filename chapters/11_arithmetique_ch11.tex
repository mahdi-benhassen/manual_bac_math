\chapter{Arithmétique dans $\mathbb{Z}$}

\section{Rappels Théoriques}

\subsection{Divisibilité et Division Euclidienne}
\begin{itemize}
    \item $b$ divise $a$ ($b|a$) s'il existe $k \in \mathbb{Z}$ tel que $a = bk$.
    \item \textbf{Division Euclidienne :} Pour tout $(a, b) \in \mathbb{Z} \times \mathbb{N}^*$, il existe un unique couple $(q, r)$ tel que :
    \[ a = bq + r \quad \text{avec } 0 \leq r < b \]
\end{itemize}

\subsection{Congruences}
$a \equiv b \pmod n \iff n | (a-b)$.
\textbf{Propriétés :} Compatibilité avec l'addition, la multiplication et les puissances.
Si $a \equiv b \pmod n$, alors $a^k \equiv b^k \pmod n$.

\subsection{PGCD et PPCM}
\begin{itemize}
    \item \textbf{Algorithme d'Euclide :} $PGCD(a, b) = PGCD(b, r)$ où $r$ est le reste de la division de $a$ par $b$.
    \item $PGCD(a, b) \times PPCM(a, b) = |ab|$.
\end{itemize}

\subsection{Théorèmes Fondamentaux}
\begin{itemize}
    \item \textbf{Théorème de Bézout :} $PGCD(a, b) = d \iff \exists (u, v) \in \mathbb{Z}^2, au + bv = d$.
    \textit{Corollaire :} $a$ et $b$ sont premiers entre eux $\iff \exists (u, v), au + bv = 1$.
    \item \textbf{Théorème de Gauss :} Si $a | bc$ et $PGCD(a, b) = 1$, alors $a | c$.
    \item \textbf{Petit Théorème de Fermat :} Si $p$ est premier et $p$ ne divise pas $a$, alors $a^{p-1} \equiv 1 \pmod p$.
    (Ou pour tout $a$, $a^p \equiv a \pmod p$).
\end{itemize}

\section{Exercices de Compréhension}

\begin{rappel}
\textbf{Objectif :} Utiliser l'algorithme d'Euclide et résoudre une congruence simple.
\end{rappel}

\textbf{Exercice 11.1 : PGCD}
Calculer le PGCD de 323 et 221.

\begin{correction}
Algorithme d'Euclide :
$323 = 221 \times 1 + 102$
$221 = 102 \times 2 + 17$
$102 = 17 \times 6 + 0$
Le dernier reste non nul est 17.
$PGCD(323, 221) = 17$.
\end{correction}

\textbf{Exercice 11.2 : Congruence}
Déterminer le reste de la division euclidienne de $3^{2026}$ par 7.

\begin{correction}
Regardons les puissances de 3 modulo 7 :
$3^0 \equiv 1 \pmod 7$
$3^1 \equiv 3 \pmod 7$
$3^2 \equiv 2 \pmod 7$
$3^3 \equiv 6 \equiv -1 \pmod 7$
$3^4 \equiv -3 \equiv 4 \pmod 7$
$3^5 \equiv 12 \equiv 5 \pmod 7$
$3^6 \equiv 15 \equiv 1 \pmod 7$. (Cycle de longueur 6).
On divise l'exposant par 6 :
$2026 = 6 \times 337 + 4$.
$3^{2026} \equiv 3^4 \pmod 7 \equiv 4 \pmod 7$.
Le reste est 4.
\end{correction}

\section{Exercices Type Bac}

\begin{exobac}
\textbf{Sujet : Équation diophantienne}

1. On considère l'équation $(E) : 11x - 7y = 1$ où $x, y \in \mathbb{Z}$.
   a) Justifier que cette équation admet des solutions.
   b) Vérifier que le couple $(2, 3)$ est une solution particulière.
   c) Résoudre l'équation $(E)$ dans $\mathbb{Z}^2$.

2. Soit $n$ un entier naturel. On pose $a = 11n + 2$ et $b = 7n + 1$.
   Montrer que $PGCD(a, b)$ divise 3.
   Pour quelles valeurs de $n$, $PGCD(a, b) = 3$ ?
\end{exobac}

\begin{correction}
\textbf{1. Équation diophantienne}
a) $PGCD(11, 7) = 1$ car 11 et 7 sont premiers. D'après le théorème de Bézout, l'équation $11x - 7y = 1$ admet des solutions.
b) $11(2) - 7(3) = 22 - 21 = 1$. $(2, 3)$ est bien solution.
c) On a :
$11x - 7y = 1$
$11(2) - 7(3) = 1$
Par soustraction : $11(x-2) - 7(y-3) = 0 \iff 11(x-2) = 7(y-3)$.
7 divise $11(x-2)$ et $PGCD(7, 11)=1$, donc d'après Gauss, 7 divise $x-2$.
$x - 2 = 7k \implies x = 7k + 2$.
En remplaçant : $11(7k) = 7(y-3) \implies 11k = y-3 \implies y = 11k + 3$.

\begin{rappel}
\textbf{Attention à la Réciproque !}
Lors de la résolution d'une équation diophantienne $ax+by=c$, après avoir exprimé $x$ et $y$ en fonction d'un paramètre $k$ (condition nécessaire), il est \textbf{impératif} de vérifier que ces solutions satisfont bien l'équation de départ (condition suffisante).
\end{rappel}

\textbf{Réciproque (Vérification) :}
$11(7k+2) - 7(11k+3) = 77k + 22 - 77k - 21 = 1$.
L'équation est vérifiée pour tout $k \in \mathbb{Z}$.
Ainsi, l'ensemble des solutions est :
$S = \{ (7k+2, 11k+3), k \in \mathbb{Z} \}$.

\textbf{2. PGCD}
Soit $d = PGCD(a, b)$.
$d$ divise toute combinaison linéaire de $a$ et $b$.
Cherchons à éliminer $n$ :
$7a - 11b = 7(11n+2) - 11(7n+1) = 77n + 14 - 77n - 11 = 3$.
Donc $d$ divise 3.
Les diviseurs positifs de 3 sont 1 et 3. Donc $d \in \{1, 3\}$.

$d = 3 \iff 3 | a \iff 11n + 2 \equiv 0 \pmod 3$.
$11 \equiv 2 \pmod 3$, donc $2n + 2 \equiv 0 \pmod 3 \iff 2n \equiv -2 \equiv 1 \pmod 3$.
Multiplions par 2 (inverse de 2 mod 3 car $2 \times 2 = 4 \equiv 1$) :
$4n \equiv 2 \pmod 3 \implies n \equiv 2 \pmod 3$.
Donc $n = 3k + 2$.
\end{correction}

\textbf{Exercice 11.3 : Puissances et Congruences}
1. Montrer que pour tout entier naturel $n$, $2^{3n} - 1$ est un multiple de 7.
2. En déduire que $2^{3n+1} - 2$ et $2^{3n+2} - 4$ sont des multiples de 7.
3. Déterminer le reste de la division par 7 de $2^{2026}$.

\begin{correction}
1. $2^3 = 8 \equiv 1 \pmod 7$.
Donc $(2^3)^n \equiv 1^n \pmod 7 \implies 2^{3n} \equiv 1 \pmod 7$.
Donc $2^{3n} - 1 \equiv 0 \pmod 7$, c'est un multiple de 7.

2. $2^{3n+1} - 2 = 2(2^{3n} - 1)$. Comme $7 | (2^{3n}-1)$, alors $7 | 2(2^{3n}-1)$.
De même, $2^{3n+2} - 4 = 4(2^{3n} - 1)$, divisible par 7.

3. $2026 = 3 \times 675 + 1$.
$2^{2026} = 2^{3 \times 675 + 1} = (2^3)^{675} \times 2^1 \equiv 1^{675} \times 2 \equiv 2 \pmod 7$.
Le reste est 2.
\end{correction}

\textbf{Exercice 11.4 : Vrai ou Faux ?}
Répondre par Vrai ou Faux en justifiant la réponse.
\begin{enumerate}
    \item Si $a \equiv b \pmod n$, alors $a^2 \equiv b^2 \pmod{n^2}$.
    \item L'équation $6x + 10y = 3$ admet des solutions dans $\mathbb{Z}^2$.
\end{enumerate}

\begin{correction}
\begin{enumerate}
    \item \textbf{Faux.} Contre-exemple : $4 \equiv 1 \pmod 3$. $4^2 = 16$ et $1^2 = 1$. Or $16 \not\equiv 1 \pmod 9$ (car $16 = 9 \times 1 + 7$).
    \item \textbf{Faux.} $PGCD(6, 10) = 2$. Or 2 ne divise pas 3. D'après le théorème de Bézout (conséquence), l'équation n'a pas de solution entière.
\end{enumerate}
\end{correction}

\autoeval{
Je sais utiliser l'algorithme d'Euclide & & \\ \hline
Je sais appliquer le théorème de Bézout et Gauss & & \\ \hline
Je sais résoudre une équation $ax+by=c$ avec la réciproque & & \\ \hline
Je sais manipuler les congruences pour trouver un reste & & \\
}

\section{Exercices Type Bac}

\textbf{2. PGCD dépendant de n}
Soit $d = PGCD(a, b)$.
$d$ divise toute combinaison linéaire de $a$ et $b$.
Cherchons à éliminer $n$.
$7a - 11b = 7(11n+2) - 11(7n+1) = 77n + 14 - 77n - 11 = 3$.
Donc $d$ divise 3.
Les diviseurs de 3 sont 1 et 3. Donc $d \in \{1, 3\}$.

$d = 3 \iff 3 | a \iff 11n + 2 \equiv 0 \pmod 3$.
$11 \equiv 2 \pmod 3 \implies 2n + 2 \equiv 0 \pmod 3$.
$2n \equiv -2 \equiv 1 \pmod 3$.
Multiplions par 2 (inverse de 2 mod 3) :
$4n \equiv 2 \pmod 3 \implies n \equiv 2 \pmod 3$.
Donc $PGCD(a, b) = 3$ si et seulement si $n = 3k + 2$.
Sinon, $PGCD(a, b) = 1$.
\end{correction}
