\chapter*{Introduction \& Méthodologie}

\section{Structure de l'épreuve}
L'épreuve de Mathématiques au Baccalauréat (Section Maths) dure \textbf{4 heures}. C'est une épreuve d'endurance et de précision.

\subsection*{Gestion du temps conseillée}
\begin{itemize}
    \item \textbf{Lecture du sujet (10 min)} : Repérer les exercices faciles, identifier les thèmes.
    \item \textbf{Exercice d'Analyse (1h30 - 1h45)} : C'est le plus gros morceau (souvent 7 à 9 points).
    \item \textbf{Exercice de Géométrie / Complexes (1h)} : 4 à 5 points.
    \item \textbf{Exercice Arithmétique / Proba (45 min)} : 3 à 4 points.
    \item \textbf{QCM / Vrai-Faux (20 min)} : S'il y en a un (3 points).
    \item \textbf{Relecture (10 min)} : Vérifier les calculs, l'orthographe et la numérotation.
\end{itemize}

\section{Les Commandements de la Rédaction}
\begin{enumerate}
    \item \textbf{Justifier toujours} : Une réponse sans justification vaut 0 (sauf mention contraire).
    \item \textbf{Citer les théorèmes} : "D'après le Théorème des Valeurs Intermédiaires...".
    \item \textbf{Encadrer les résultats} : Facilite la tâche du correcteur.
    \item \textbf{Ne pas bloquer} : Si vous ne trouvez pas la réponse, admettez le résultat donné dans l'énoncé pour continuer les questions suivantes.
    \item \textbf{Soigner la copie} : Écriture lisible, figures propres (au crayon, repassées si sûr).
\end{enumerate}

\section{Utilisation de la Calculatrice}
La calculatrice est un outil de \textit{vérification}, pas de \textit{preuve}.
\begin{itemize}
    \item \textbf{Intégrales} : Vérifiez vos calculs de primitives en calculant l'intégrale définie numériquement.
    \item \textbf{Matrices} : Vérifiez les produits matriciels et les inversions.
    \item \textbf{Complexes} : Vérifiez les passages forme algébrique $\leftrightarrow$ forme exponentielle.
\end{itemize}
\textbf{Attention :} Écrire "D'après la calculatrice..." n'est généralement pas accepté comme justification complète.
