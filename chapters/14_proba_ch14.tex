\chapter{Statistiques à deux variables}

\section{Rappels Théoriques}

\subsection{Série Statistique Double}
On étudie deux caractères $X$ et $Y$ sur une même population.
On dispose de $n$ couples $(x_i, y_i)$.
On représente cette série par un \textbf{nuage de points} $M_i(x_i, y_i)$ dans un repère orthogonal.

\subsection{Point Moyen}
Le point moyen $G(\bar{x}, \bar{y})$ est le centre de gravité du nuage.
\[ \bar{x} = \frac{1}{n} \sum_{i=1}^n x_i \quad ; \quad \bar{y} = \frac{1}{n} \sum_{i=1}^n y_i \]

\subsection{Covariance et Corrélation}
\begin{itemize}
    \item \textbf{Covariance :} $Cov(X, Y) = \frac{1}{n} \sum_{i=1}^n (x_i - \bar{x})(y_i - \bar{y}) = (\frac{1}{n} \sum x_i y_i) - \bar{x}\bar{y}$.
    \item \textbf{Coefficient de corrélation linéaire :}
    \[ r = \frac{Cov(X, Y)}{\sigma_X \sigma_Y} \]
    où $\sigma_X$ et $\sigma_Y$ sont les écarts-types marginaux.
    \textbf{Interprétation :} $-1 \leq r \leq 1$.
    \begin{itemize}
        \item Si $|r|$ est proche de 1 (en général $|r| \geq 0.85$ ou $0.9$), il y a une \textbf{forte corrélation linéaire}. L'ajustement affine est alors justifié et de bonne qualité.
        \item Si $|r|$ est proche de 0, les variables ne sont pas linéairement corrélées.
    \end{itemize}
\end{itemize}

\subsection{Ajustement Affine}
Si le nuage a une forme allongée, on peut chercher une droite $D : y = ax + b$ qui "résume" le nuage.
\textbf{Méthode des moindres carrés :}
La droite de régression de $Y$ en $X$ a pour équation $y = ax + b$ avec :
\[ a = \frac{Cov(X, Y)}{V(X)} \quad ; \quad b = \bar{y} - a\bar{x} \]
Cette droite passe par le point moyen $G$.

\section{Exercices de Compréhension}

\begin{rappel}
\textbf{Objectif :} Calculer les paramètres d'une série double et l'équation de la droite de régression.
\end{rappel}

\textbf{Exercice 14.1 : Calculs de base}
Soit la série double :
\begin{center}
\begin{tabular}{|c|c|c|c|}
\hline
$x_i$ & 1 & 2 & 3 \\
\hline
$y_i$ & 2 & 4 & 5 \\
\hline
\end{tabular}
\end{center}
1. Calculer $\bar{x}$ et $\bar{y}$.
2. Calculer $Cov(X, Y)$.
3. Déterminer l'équation de la droite de régression de $Y$ en $X$.

\begin{correction}
1. $\bar{x} = \frac{1+2+3}{3} = 2$.
$\bar{y} = \frac{2+4+5}{3} = \frac{11}{3} \approx 3.67$.

2. Moyenne des produits $\overline{xy} = \frac{1\times2 + 2\times4 + 3\times5}{3} = \frac{2+8+15}{3} = \frac{25}{3}$.
$Cov(X, Y) = \overline{xy} - \bar{x}\bar{y} = \frac{25}{3} - 2 \times \frac{11}{3} = \frac{25 - 22}{3} = \frac{3}{3} = 1$.

3. Variance de X : $V(X) = \frac{1^2+2^2+3^2}{3} - \bar{x}^2 = \frac{1+4+9}{3} - 4 = \frac{14}{3} - \frac{12}{3} = \frac{2}{3}$.
Coefficient directeur $a = \frac{Cov(X, Y)}{V(X)} = \frac{1}{2/3} = \frac{3}{2} = 1.5$.
Ordonnée à l'origine $b = \bar{y} - a\bar{x} = \frac{11}{3} - \frac{3}{2}(2) = \frac{11}{3} - 3 = \frac{2}{3}$.
Droite : $y = 1.5x + \frac{2}{3}$.
\end{correction}

\section{Exercices Type Bac}

\begin{exobac}
\textbf{Sujet : Ajustement exponentiel}

Une entreprise étudie l'évolution de son chiffre d'affaires $y_i$ (en milliers de dinars) en fonction du rang de l'année $x_i$.
\begin{center}
\begin{tabular}{|c|c|c|c|c|c|}
\hline
Année & 1 & 2 & 3 & 4 & 5 \\
\hline
$y_i$ & 18 & 22 & 27 & 33 & 40 \\
\hline
\end{tabular}
\end{center}
Le nuage de points montre une croissance non linéaire. On pose $z_i = \ln(y_i)$.
\textit{(Voir Chapitre 3 sur la fonction Logarithme pour les propriétés de $\ln$).}

1. Dresser le tableau des valeurs $z_i$ (arrondies à $10^{-2}$).
2. Calculer le coefficient de corrélation linéaire $r$ entre $x$ et $z$. Un ajustement affine est-il justifié ?
3. Déterminer l'équation de la droite de régression de $z$ en $x$ ($z = ax+b$).
4. En déduire une relation de la forme $y = A e^{Bx}$.
5. Estimer le chiffre d'affaires prévisionnel pour l'année 7.
\end{exobac}

\begin{correction}
\textbf{1. Tableau des $z_i = \ln(y_i)$}
$z_1 = \ln(18) \approx 2.89$.
$z_2 = \ln(22) \approx 3.09$.
$z_3 = \ln(27) \approx 3.30$.
$z_4 = \ln(33) \approx 3.50$.
$z_5 = \ln(40) \approx 3.69$.

\textbf{2. Corrélation}
À la calculatrice :
$\bar{x} = 3$, $\bar{z} \approx 3.29$.
$r \approx 0.999$.
Comme $r$ est très proche de 1, un ajustement affine entre $x$ et $z$ est \textbf{très justifié}.

\textbf{3. Droite de régression $z$ en $x$}
À la calculatrice :
$a \approx 0.20$.
$b \approx 2.69$.
$z = 0.20x + 2.69$.

\textbf{4. Relation $y$ et $x$}
On a $\ln(y) = 0.20x + 2.69$.
Donc $y = e^{0.20x + 2.69} = e^{2.69} \times e^{0.20x}$.
$e^{2.69} \approx 14.73$.
Donc $y \approx 14.73 e^{0.20x}$.

\textbf{5. Prévision Année 7}
Pour $x = 7$ :
$y = 14.73 \times e^{0.20 \times 7} = 14.73 \times e^{1.4}$.
$y \approx 14.73 \times 4.055 \approx 59.7$.
Le chiffre d'affaires prévisionnel est d'environ \textbf{59 700 dinars}.
\end{correction}

\textbf{Exercice 14.3 : Vrai ou Faux ?}
Répondre par Vrai ou Faux en justifiant.
\begin{enumerate}
    \item Si le coefficient de corrélation est $r = -0.95$, alors il y a une forte corrélation linéaire entre les variables.
    \item La droite de régression passe toujours par le point moyen $G(\bar{x}, \bar{y})$.
\end{enumerate}

\begin{correction}
\begin{enumerate}
    \item \textbf{Vrai.} $|r| = 0.95$ est proche de 1. Le signe négatif indique juste que la relation est décroissante.
    \item \textbf{Vrai.} C'est une propriété fondamentale de la méthode des moindres carrés.
\end{enumerate}
\end{correction}

\autoeval{
Je sais représenter un nuage de points et calculer le point moyen & & \\ \hline
Je sais calculer la covariance et le coefficient de corrélation & & \\ \hline
Je sais déterminer la droite de régression (Moyenne, Moindres Carrés) & & \\ \hline
Je sais effectuer un changement de variable pour un ajustement non linéaire & & \\
}
