\chapter{Variables Aléatoires et Lois}

\section{Rappels Théoriques}

\subsection{Variable Aléatoire Discrète}
Une variable aléatoire $X$ associe un réel à chaque issue de l'univers $\Omega$.
\begin{itemize}
    \item \textbf{Loi de probabilité :} On donne les valeurs $x_i$ prises par $X$ et les probabilités $p_i = P(X=x_i)$. (Avec $\sum p_i = 1$).
    \item \textbf{Espérance :} $E(X) = \sum x_i p_i$. (Moyenne théorique).
    \item \textbf{Variance :} $V(X) = \sum p_i (x_i - E(X))^2 = E(X^2) - (E(X))^2$.
    \item \textbf{Écart-type :} $\sigma(X) = \sqrt{V(X)}$.
\end{itemize}

\subsection{Loi Binomiale $\mathcal{B}(n, p)$}
On répète $n$ fois une épreuve de Bernoulli (Succès $p$, Échec $q=1-p$) de manière identique et indépendante.
$X$ compte le nombre de succès.
\[ P(X = k) = C_n^k p^k (1-p)^{n-k} \quad \text{pour } k \in \{0, \dots, n\} \]
\begin{itemize}
    \item Espérance : $E(X) = np$.
    \item Variance : $V(X) = np(1-p)$.
\end{itemize}

\subsection{Lois à Densité (Continues)}
$X$ prend ses valeurs dans un intervalle $I$. La probabilité est définie par une fonction densité $f \geq 0$ telle que $\int_I f(t) dt = 1$.
\[ P(a \leq X \leq b) = \int_a^b f(t) dt \]

\textbf{Loi Uniforme sur $[a, b]$ :}
Densité constante $f(x) = \frac{1}{b-a}$ sur $[a, b]$.
\[ P(c \leq X \leq d) = \frac{d-c}{b-a} \]
Espérance : $E(X) = \frac{a+b}{2}$.

\textbf{Loi Exponentielle de paramètre $\lambda > 0$ :}
Densité sur $[0, +\infty[$ : $f(t) = \lambda e^{-\lambda t}$.
\[ P(X \leq t) = \int_0^t \lambda e^{-\lambda x} dx = [-e^{-\lambda x}]_0^t = 1 - e^{-\lambda t} \]
\[ P(X > t) = e^{-\lambda t} \]
\begin{itemize}
    \item Espérance (durée de vie moyenne) : $E(X) = \frac{1}{\lambda}$.
    \item \textbf{Absence de mémoire :} $P(X > t+s | X > t) = P(X > s)$.
\end{itemize}

\section{Exercices de Compréhension}

\begin{rappel}
\textbf{Objectif :} Calculer l'espérance d'une loi binomiale et une probabilité exponentielle.
\end{rappel}

\textbf{Exercice 13.1 : Loi Binomiale}
Un tireur atteint sa cible avec une probabilité $p=0.8$. Il tire 10 fois.
Soit $X$ le nombre de succès.
1. Quelle est la loi de $X$ ?
2. Calculer l'espérance de $X$.
3. Calculer la probabilité d'atteindre exactement 9 fois la cible.

\begin{correction}
1. Répétition identique et indépendante (supposée). $X$ suit $\mathcal{B}(10, 0.8)$.
2. $E(X) = n \times p = 10 \times 0.8 = 8$. (En moyenne, il touche 8 fois).
3. $P(X=9) = C_{10}^9 (0.8)^9 (0.2)^1 = 10 \times 0.8^9 \times 0.2 \approx 0.268$.
\end{correction}

\textbf{Exercice 13.2 : Loi Exponentielle}
La durée de vie d'un composant suit une loi exponentielle de moyenne 1000 heures.
Calculer la probabilité qu'il dure plus de 1000 heures.

\begin{correction}
$E(X) = 1/\lambda = 1000 \implies \lambda = 0.001$.
$P(X > 1000) = e^{-0.001 \times 1000} = e^{-1} \approx 0.368$.
\end{correction}

\section{Exercices Type Bac}

\begin{exobac}
\textbf{Sujet : Loi Exponentielle et Probabilités Conditionnelles}

La durée de vie (en années) d'un appareil électronique est une variable aléatoire $T$ qui suit une loi exponentielle de paramètre $\lambda$.
On sait que la probabilité que l'appareil tombe en panne avant 1 an est $P(T \leq 1) = 0.18$.

1. Déterminer la valeur exacte de $\lambda$, puis une valeur approchée à $10^{-3}$ près.
2. On prendra $\lambda = 0.2$ pour la suite.
   Calculer la probabilité que l'appareil fonctionne plus de 5 ans.
3. Sachant que l'appareil a fonctionné 3 ans, quelle est la probabilité qu'il fonctionne encore au moins 2 ans ?
4. On considère un lot de 5 appareils indépendants. Quelle est la probabilité qu'au moins un d'entre eux fonctionne plus de 5 ans ?
\end{exobac}

\begin{correction}
\textbf{1. Calcul de $\lambda$}
$P(T \leq 1) = 1 - e^{-\lambda(1)} = 1 - e^{-\lambda}$.
On a $1 - e^{-\lambda} = 0.18 \iff e^{-\lambda} = 0.82$.
$\iff -\lambda = \ln(0.82) \iff \lambda = -\ln(0.82)$.
$\lambda \approx 0.198$.

\textbf{2. Durée de vie $> 5$ ans}
$P(T > 5) = e^{-0.2 \times 5} = e^{-1} \approx 0.368$.

\textbf{3. Durée de vie sans vieillissement}
On cherche $P(T > 3+2 | T > 3)$.
D'après la propriété d'absence de mémoire :
$P(T > 5 | T > 3) = P(T > 2)$.
$P(T > 2) = e^{-0.2 \times 2} = e^{-0.4} \approx 0.670$.
Cela signifie que le fait qu'il ait déjà duré 3 ans n'influence pas sa probabilité de durer 2 ans \textit{de plus}.

\textbf{4. Lot de 5 appareils}
Soit $Y$ le nombre d'appareils fonctionnant plus de 5 ans parmi les 5.
$Y$ suit une loi binomiale $\mathcal{B}(5, p)$ avec $p = P(T > 5) = e^{-1}$.
On cherche $P(Y \geq 1) = 1 - P(Y = 0)$.
$P(Y = 0) = C_5^0 p^0 (1-p)^5 = (1 - e^{-1})^5$.
Donc $P(Y \geq 1) = 1 - (1 - e^{-1})^5 \approx 1 - (0.632)^5 \approx 1 - 0.10 \approx 0.90$.
\end{correction}

\textbf{Exercice 13.3 : Probabilités Totales (Arbre)}
Une usine fabrique des pièces. 2\% sont défectueuses.
On dispose d'un test de contrôle.
- Si la pièce est bonne ($B$), le test l'accepte ($A$) avec probabilité 0.96.
- Si la pièce est défectueuse ($D$), le test la refuse ($R$) avec probabilité 0.98.
1. Faire un arbre pondéré.
2. Quelle est la probabilité qu'une pièce soit acceptée ?
3. Sachant qu'une pièce est acceptée, quelle est la probabilité qu'elle soit défectueuse (Erreur du test) ?

\begin{correction}
\textbf{1. Arbre}
$P(D) = 0.02 \implies P(B) = 0.98$.
$P(A|B) = 0.96 \implies P(R|B) = 0.04$.
$P(R|D) = 0.98 \implies P(A|D) = 0.02$.

\textbf{2. Probabilité d'acceptation $P(A)$}
D'après la formule des probabilités totales :
$P(A) = P(B) \times P(A|B) + P(D) \times P(A|D)$
$P(A) = 0.98 \times 0.96 + 0.02 \times 0.02$
$P(A) = 0.9408 + 0.0004 = 0.9412$.

\textbf{3. Probabilité $P(D|A)$ (Théorème de Bayes)}
$P(D|A) = \frac{P(D \cap A)}{P(A)} = \frac{P(D) \times P(A|D)}{P(A)}$.
$P(D|A) = \frac{0.0004}{0.9412} \approx 0.000425$.
La probabilité est très faible (environ 0.04\%).
\end{correction}

\textbf{Exercice 13.4 : Vrai ou Faux ?}
Répondre par Vrai ou Faux en justifiant.
\begin{enumerate}
    \item Si $X$ suit la loi uniforme sur $[0, 10]$, alors $P(X \geq 5) = 0.5$.
    \item L'espérance d'une variable aléatoire binomiale $\mathcal{B}(10, 0.5)$ est 5.
\end{enumerate}

\begin{correction}
\begin{enumerate}
    \item \textbf{Vrai.} $P(X \geq 5) = \frac{10-5}{10-0} = \frac{5}{10} = 0.5$.
    \item \textbf{Vrai.} $E(X) = np = 10 \times 0.5 = 5$.
\end{enumerate}
\end{correction}

\autoeval{
Je sais calculer l'espérance et la variance d'une variable aléatoire & & \\ \hline
Je reconnais une loi binomiale et sais appliquer sa formule & & \\ \hline
Je sais faire des calculs avec la loi exponentielle (durée de vie) & & \\ \hline
Je sais utiliser la propriété d'absence de mémoire & & \\
}
