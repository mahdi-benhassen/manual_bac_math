\chapter{Suites Réelles}

\section{Rappels Théoriques}

\subsection{Raisonnement par Récurrence}
Pour démontrer qu'une propriété $P(n)$ est vraie pour tout $n \geq n_0$ :
\begin{enumerate}
    \item \textbf{Initialisation :} Vérifier que $P(n_0)$ est vraie.
    \item \textbf{Hérédité :} Supposer que $P(k)$ est vraie pour un certain entier $k \geq n_0$, et montrer que $P(k+1)$ est vraie.
    \item \textbf{Conclusion :} Par récurrence, $P(n)$ est vraie pour tout $n \geq n_0$.
\end{enumerate}

\subsection{Convergence et Monotonie}
\begin{itemize}
    \item Une suite \textbf{croissante et majorée} est convergente.
    \item Une suite \textbf{décroissante et minorée} est convergente.
    \item \textbf{Théorème des Gendarmes :} Si $v_n \leq u_n \leq w_n$ et si $\lim v_n = \lim w_n = L$, alors $\lim u_n = L$.
    \item \textbf{Inégalité géométrique :} Si $|u_n - L| \leq k^n |u_0 - L|$ avec $0 < k < 1$, alors $\lim u_n = L$.
\end{itemize}

\subsection{Suites Récurrentes $u_{n+1} = f(u_n)$}
Si $f$ est continue sur un intervalle $I$ et si la suite $(u_n)$ converge vers $L \in I$, alors $L$ est solution de l'équation $f(x) = x$ (Point fixe).

\subsection{Suites Arithmétiques et Géométriques}
\begin{center}
\begin{tabular}{|c|c|c|}
\hline
& \textbf{Arithmétique} & \textbf{Géométrique} \\
\hline
Définition & $u_{n+1} = u_n + r$ & $u_{n+1} = q \times u_n$ \\
\hline
Terme général & $u_n = u_0 + nr$ & $u_n = u_0 \times q^n$ \\
\hline
Somme $S_n$ & $\frac{(n+1)(u_0 + u_n)}{2}$ & $u_0 \frac{1 - q^{n+1}}{1 - q}$ ($q \neq 1$) \\
\hline
Convergence & Vers $\pm \infty$ (si $r \neq 0$) & Vers 0 si $|q| < 1$, Diverge sinon \\
\hline
\end{tabular}
\end{center}

\subsection{Suites Adjacentes}
Deux suites $(u_n)$ et $(v_n)$ sont adjacentes si :
\begin{itemize}
    \item L'une est croissante, l'autre est décroissante.
    \item $\lim_{n \to +\infty} (v_n - u_n) = 0$.
\end{itemize}
\textbf{Théorème :} Si deux suites sont adjacentes, alors elles convergent et ont la même limite.

\section{Exercices de Compréhension}

\begin{rappel}
\textbf{Objectif :} Étudier la monotonie et calculer des limites simples.
\end{rappel}

\textbf{Exercice 6.1 : Monotonie}
Soit $u_n = \frac{2n+1}{n+2}$ pour $n \in \mathbb{N}$. Étudier le sens de variation de $(u_n)$.

\begin{correction}
Soit $f(x) = \frac{2x+1}{x+2}$ sur $[0, +\infty[$.
$f'(x) = \frac{2(x+2) - 1(2x+1)}{(x+2)^2} = \frac{3}{(x+2)^2} > 0$.
La fonction $f$ est strictement croissante, donc la suite $(u_n)$ est \textbf{croissante}.
\textit{Autre méthode :} Calculer $u_{n+1} - u_n$.
\end{correction}

\textbf{Exercice 6.2 : Limites}
Calculer $\lim_{n \to +\infty} \frac{3^n - 2^n}{3^n + 2^n}$.

\begin{correction}
Factorisons par le terme dominant $3^n$ :
\[ \frac{3^n(1 - (2/3)^n)}{3^n(1 + (2/3)^n)} = \frac{1 - (2/3)^n}{1 + (2/3)^n} \]
Comme $-1 < 2/3 < 1$, $\lim (2/3)^n = 0$.
Donc la limite est $\frac{1-0}{1+0} = 1$.
\end{correction}

\section{Exercices Type Bac}

\begin{exobac}
\textbf{Sujet : Suites adjacentes et nombre $e$}

On considère les suites $(u_n)$ et $(v_n)$ définies pour $n \geq 1$ par :
\[ u_n = \sum_{k=0}^n \frac{1}{k!} = 1 + \frac{1}{1!} + \frac{1}{2!} + \dots + \frac{1}{n!} \]
\[ v_n = u_n + \frac{1}{n \cdot n!} \]
\begin{enumerate}
    \item Montrer que la suite $(u_n)$ est croissante.
    \item Montrer que la suite $(v_n)$ est décroissante.
    \item Montrer que les suites $(u_n)$ et $(v_n)$ sont adjacentes.
    \item Que peut-on en déduire ? (Leur limite commune est le nombre $e$).
\end{enumerate}
\end{exobac}

\begin{correction}
\textbf{1. Croissance de $(u_n)$}
$u_{n+1} - u_n = \frac{1}{(n+1)!}$.
Comme $(n+1)! > 0$, alors $u_{n+1} - u_n > 0$.
La suite $(u_n)$ est \textbf{strictement croissante}.

\textbf{2. Décroissance de $(v_n)$}
$v_{n+1} - v_n = u_{n+1} + \frac{1}{(n+1)(n+1)!} - (u_n + \frac{1}{n \cdot n!})$
$= (u_{n+1} - u_n) + \frac{1}{(n+1)(n+1)!} - \frac{1}{n \cdot n!}$
$= \frac{1}{(n+1)!} + \frac{1}{(n+1)(n+1)!} - \frac{1}{n \cdot n!}$
Factorisons par $\frac{1}{n!(n+1)}$ :
$= \frac{1}{n!(n+1)} [1 + \frac{1}{n+1} - \frac{n+1}{n}]$
$= \frac{1}{(n+1)!} [1 + \frac{1}{n+1} - (1 + \frac{1}{n})]$
$= \frac{n(n+1) + n - (n+1)^2}{n(n+1)(n+1)!} = \frac{n^2 + n + n - (n^2 + 2n + 1)}{n(n+1)(n+1)!} = \frac{-1}{n(n+1)(n+1)!}$.
C'est négatif pour tout $n \geq 1$.
Donc $(v_n)$ est \textbf{strictement décroissante}.

\textbf{3. Adjacentes}
$v_n - u_n = \frac{1}{n \cdot n!}$.
$\lim_{n \to +\infty} n \cdot n! = +\infty$, donc $\lim (v_n - u_n) = 0$.
De plus, $u_n$ croissante et $v_n$ décroissante.
Elles sont donc \textbf{adjacentes}.

\textbf{4. Conclusion}
Elles convergent vers la même limite. Cette limite est $e \approx 2.718$.
\end{correction}

\textbf{Exercice 6.4 : Suite récurrente et convergence}
Soit la suite $(u_n)$ définie par $u_0 = 1$ et $u_{n+1} = \sqrt{2u_n + 3}$.
\begin{enumerate}
    \item Montrer par récurrence que pour tout $n \in \mathbb{N}$, $0 \leq u_n \leq 3$.
    \item Montrer que $(u_n)$ est croissante.
    \item En déduire qu'elle converge et calculer sa limite.
\end{enumerate}

\begin{correction}
Soit $f(x) = \sqrt{2x+3}$. $f$ est croissante sur $[-1.5, +\infty[$.
\textbf{1. Récurrence}
$P(n) : 0 \leq u_n \leq 3$.
$n=0 : u_0 = 1 \in [0, 3]$. Vrai.
Hérédité : Supposons $0 \leq u_n \leq 3$.
$f$ croissante \implies f(0) \leq f(u_n) \leq f(3).
$\sqrt{3} \leq u_{n+1} \leq \sqrt{9}=3$.
Or $0 \leq \sqrt{3}$, donc $0 \leq u_{n+1} \leq 3$.
Vrai pour tout $n$.

\textbf{2. Monotonie}
$u_{n+1} - u_n = \sqrt{2u_n+3} - u_n = \frac{2u_n+3 - u_n^2}{\sqrt{2u_n+3} + u_n}$.
Signe de $-u_n^2 + 2u_n + 3$. Racines : $-1$ et $3$.
Sur $[0, 3]$, ce trinôme est positif.
Donc $u_{n+1} \geq u_n$. Croissante.

\textbf{3. Limite}
Croissante et majorée par 3, donc converge vers $L \in [0, 3]$.
$L = \sqrt{2L+3} \implies L^2 - 2L - 3 = 0 \implies L=3$ ou $L=-1$.
Comme $u_n \geq 0$, $L=3$.
\end{correction}

\textbf{Exercice 6.5 : Vrai ou Faux ?}
Répondre par Vrai ou Faux en justifiant.
\begin{enumerate}
    \item Si $(u_n)$ est bornée, alors elle est convergente.
    \item Si $(u_n)$ est croissante et non majorée, alors $\lim u_n = +\infty$.
\end{enumerate}

\begin{correction}
\begin{enumerate}
    \item \textbf{Faux.} Contre-exemple : $u_n = (-1)^n$. Bornée par -1 et 1, mais diverge.
    \item \textbf{Vrai.} C'est un théorème fondamental du cours sur les limites de suites monotones.
\end{enumerate}
\end{correction}

\autoeval{
Je sais faire un raisonnement par récurrence & & \\ \hline
Je sais utiliser le théorème des gendarmes & & \\ \hline
Je sais étudier une suite récurrente $u_{n+1}=f(u_n)$ & & \\ \hline
Je connais les formules des suites géométriques & & \\
}

\section{Exercices Type Bac}
