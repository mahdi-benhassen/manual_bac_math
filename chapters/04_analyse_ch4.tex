\chapter{Calcul Intégral}

\section{Rappels Théoriques}

\subsection{Primitives}
\begin{definition}
Soit $f$ une fonction continue sur un intervalle $I$. On appelle primitive de $f$ sur $I$ toute fonction $F$ dérivable sur $I$ telle que $F' = f$.
\end{definition}
\textbf{Théorème :} Toute fonction continue sur un intervalle admet des primitives. Si $F$ est une primitive, alors toutes les primitives sont de la forme $F(x) + k$ ($k \in \mathbb{R}$).

\textbf{Primitives usuelles :}
\begin{center}
\begin{tabular}{|c|c|}
\hline
\textbf{Fonction $f$} & \textbf{Primitive $F$} \\
\hline
$x^n$ ($n \neq -1$) & $\frac{x^{n+1}}{n+1}$ \\
\hline
$\frac{1}{x}$ ($x > 0$) & $\ln x$ \\
\hline
$e^x$ & $e^x$ \\
\hline
$\cos x$ & $\sin x$ \\
\hline
$\sin x$ & $-\cos x$ \\
\hline
$u' u^n$ & $\frac{u^{n+1}}{n+1}$ \\
\hline
$\frac{u'}{u}$ & $\ln |u|$ \\
\hline
$u' e^u$ & $e^u$ \\
\hline
\end{tabular}
\end{center}

\subsection{Intégrale Définie}
Soit $f$ une fonction continue sur $[a, b]$ et $F$ une primitive de $f$.
\[ \int_a^b f(t) dt = [F(t)]_a^b = F(b) - F(a) \]

\textbf{Propriétés :}
\begin{itemize}
    \item \textbf{Linéarité :} $\int (\alpha f + \beta g) = \alpha \int f + \beta \int g$.
    \item \textbf{Relation de Chasles :} $\int_a^b f + \int_b^c f = \int_a^c f$.
    \item \textbf{Positivité :} Si $a \leq b$ et $f \geq 0$, alors $\int_a^b f(t) dt \geq 0$.
    \item \textbf{Ordre :} Si $a \leq b$ et $f \leq g$, alors $\int_a^b f \leq \int_a^b g$.
\end{itemize}

\subsection{Intégration par Parties (IPP)}
Soient $u$ et $v$ deux fonctions dérivables sur $[a, b]$ telles que $u'$ et $v'$ soient continues.
\[ \int_a^b u(x)v'(x) dx = [u(x)v(x)]_a^b - \int_a^b u'(x)v(x) dx \]
\textbf{Moyen mnémotechnique ALPES} pour choisir $u$ (celui qu'on dérive) :
\textbf{A}rc (Arctan...) > \textbf{L}n > \textbf{P}olynôme > \textbf{E}xponentielle > \textbf{S}inus/Cosinus.

\subsection{Calcul d'Aires et Volumes}
\begin{itemize}
    \item \textbf{Aire :} L'aire du domaine délimité par $\mathcal{C}_f$, l'axe $(Ox)$ et les droites $x=a, x=b$ est $\mathcal{A} = \int_a^b |f(x)| dx$ (en unités d'aire).
    \item \textbf{Volume :} Le volume du solide engendré par la rotation de $\mathcal{C}_f$ autour de l'axe $(Ox)$ sur $[a, b]$ est $V = \pi \int_a^b (f(x))^2 dx$ (en unités de volume).
\end{itemize}

\subsection{Fonction définie par une intégrale}
La fonction $F(x) = \int_a^x f(t) dt$ est \textbf{la} primitive de $f$ qui s'annule en $a$.
$F$ est dérivable et $F'(x) = f(x)$.

\textbf{Signe d'une intégrale sans calcul :}
Pour déterminer le signe de $F(x) = \int_{u(x)}^{v(x)} f(t) dt$, il faut étudier deux éléments :
\begin{enumerate}
    \item \textbf{Le signe de la fonction intégrée $f$ :} Si $f$ est positive sur l'intervalle d'intégration.
    \item \textbf{L'ordre des bornes :}
    \begin{itemize}
        \item Si $u(x) \leq v(x)$ (bornes dans l'ordre croissant) et $f \geq 0$, alors $F(x) \geq 0$.
        \item Si $u(x) \geq v(x)$ (bornes dans l'ordre décroissant) et $f \geq 0$, alors $F(x) \leq 0$.
    \end{itemize}
\end{enumerate}
\textit{Exemple :} $\int_2^x \sqrt{t} dt$. Si $x > 2$, bornes croissantes et $\sqrt{t} > 0 \implies$ Intégrale $> 0$. Si $x < 2$, bornes décroissantes et $\sqrt{t} > 0 \implies$ Intégrale $< 0$.

\section{Exercices de Compréhension}

\begin{rappel}
\textbf{Objectif :} Calculer des primitives simples et appliquer la formule d'IPP.
\end{rappel}

\textbf{Exercice 4.1 : Primitives}
Déterminer une primitive des fonctions suivantes :
\begin{enumerate}
    \item $f(x) = \frac{2x}{x^2+1}$
    \item $g(x) = (x+1)e^{x^2+2x}$
\end{enumerate}

\begin{correction}
\begin{enumerate}
    \item Forme $\frac{u'}{u}$ avec $u = x^2+1$. $F(x) = \ln(x^2+1)$.
    \item Forme $\frac{1}{2} u' e^u$ avec $u = x^2+2x$ donc $u' = 2x+2 = 2(x+1)$.
    $g(x) = \frac{1}{2} (2x+2) e^{x^2+2x}$.
    $G(x) = \frac{1}{2} e^{x^2+2x}$.
\end{enumerate}
\end{correction}

\textbf{Exercice 4.2 : IPP}
Calculer $I = \int_1^e x \ln x dx$.

\begin{correction}
On pose $u(x) = \ln x$ (choix "L" avant "P") et $v'(x) = x$.
Donc $u'(x) = \frac{1}{x}$ et $v(x) = \frac{x^2}{2}$.
\[ I = [\frac{x^2}{2} \ln x]_1^e - \int_1^e \frac{1}{x} \cdot \frac{x^2}{2} dx \]
\[ I = (\frac{e^2}{2} \ln e - \frac{1}{2} \ln 1) - \int_1^e \frac{x}{2} dx \]
\[ I = \frac{e^2}{2} - [\frac{x^2}{4}]_1^e \]
\[ I = \frac{e^2}{2} - (\frac{e^2}{4} - \frac{1}{4}) = \frac{2e^2 - e^2 + 1}{4} = \frac{e^2+1}{4} \]
\end{correction}

\textbf{Exercice 4.3 : Suite d'intégrales et Encadrement}
On considère la suite $(u_n)$ définie par $u_n = \int_0^1 \frac{x^n}{1+x} dx$.
1. Calculer $u_0$.
2. Étudier la monotonie de la suite $(u_n)$.
3. Montrer que pour tout $n \in \mathbb{N}$, $0 \leq u_n \leq \frac{1}{n+1}$. En déduire la limite de $(u_n)$.

\begin{correction}
1. $u_0 = \int_0^1 \frac{1}{1+x} dx = [\ln(1+x)]_0^1 = \ln 2 - \ln 1 = \ln 2$.

2. $u_{n+1} - u_n = \int_0^1 \frac{x^{n+1} - x^n}{1+x} dx = \int_0^1 \frac{x^n(x-1)}{1+x} dx$.
Sur $[0, 1]$, $x^n \geq 0$, $1+x > 0$ et $x-1 \leq 0$.
Donc l'intégrande est négatif.
Ainsi, $u_{n+1} - u_n \leq 0$, la suite est \textbf{décroissante}.

3. Sur $[0, 1]$, $1 \leq 1+x \leq 2 \implies \frac{1}{2} \leq \frac{1}{1+x} \leq 1$.
En multipliant par $x^n$ (positif) : $\frac{x^n}{2} \leq \frac{x^n}{1+x} \leq x^n$.
En intégrant :
$\int_0^1 0 dx \leq u_n \leq \int_0^1 x^n dx$ (car l'intégrande est positif).
$0 \leq u_n \leq [\frac{x^{n+1}}{n+1}]_0^1 = \frac{1}{n+1}$.
D'après le théorème des gendarmes, $\lim_{n \to +\infty} u_n = 0$.
\end{correction}

\textbf{Exercice 4.4 : Calcul de Volume} \\
Soit $f$ la fonction définie sur $[0, \pi]$ par $f(x) = \sqrt{\sin x}$.
Calculer le volume $V$ du solide engendré par la rotation de la courbe $\mathcal{C}_f$ autour de l'axe des abscisses.

\begin{correction}
La formule du volume est $V = \pi \int_a^b (f(x))^2 dx$.
Ici $a=0, b=\pi$ et $(f(x))^2 = (\sqrt{\sin x})^2 = \sin x$ (car $\sin x \geq 0$ sur $[0, \pi]$).
\[ V = \pi \int_0^\pi \sin x dx = \pi [-\cos x]_0^\pi \]
\[ V = \pi (-\cos(\pi) - (-\cos(0))) = \pi (-(-1) - (-1)) = \pi (1+1) = 2\pi \text{ unités de volume}. \]
\end{correction}

\textbf{Exercice 4.5 : Vrai ou Faux ?}
Répondre par Vrai ou Faux en justifiant.
\begin{enumerate}
    \item Si $F$ est une primitive de $f$ sur $I$, alors $F$ est dérivable sur $I$.
    \item $\int_{-1}^1 x^3 dx = 0$.
\end{enumerate}

\begin{correction}
\begin{enumerate}
    \item \textbf{Vrai.} Par définition, $F'(x) = f(x)$. Donc $F$ est dérivable.
    \item \textbf{Vrai.} La fonction $x \mapsto x^3$ est impaire et l'intervalle est symétrique par rapport à 0. (Calcul : $[\frac{x^4}{4}]_{-1}^1 = \frac{1}{4} - \frac{1}{4} = 0$).
\end{enumerate}
\end{correction}

\autoeval{
Je sais calculer une primitive simple (tableau) & & \\ \hline
Je sais faire une Intégration par Parties (IPP) & & \\ \hline
Je sais calculer une aire et un volume & & \\ \hline
Je sais encadrer une suite d'intégrales & & \\
}

\section{Exercices Type Bac}

\begin{exobac}
\textbf{Sujet : Suite d'intégrales et calcul d'aire}

Soit la suite d'intégrales $(I_n)$ définie pour $n \geq 1$ par :
\[ I_n = \int_0^1 x^n e^{-x} dx \]
\begin{enumerate}
    \item Calculer $I_1$.
    \item Montrer que pour tout $n \geq 1$, $0 \leq I_n \leq \frac{1}{n+1}$. En déduire la limite de $I_n$.
    \item À l'aide d'une intégration par parties, montrer que $I_{n+1} = (n+1)I_n - \frac{1}{e}$.
    \item Soit $f(x) = xe^{-x}$. Calculer l'aire $\mathcal{A}$ de la partie du plan délimitée par la courbe $\mathcal{C}_f$, l'axe des abscisses et les droites $x=0$ et $x=1$.
\end{enumerate}
\end{exobac}

\begin{correction}
\textbf{1. Calcul de $I_1$}
$I_1 = \int_0^1 x e^{-x} dx$.
IPP : $u(x) = x \implies u'(x) = 1$. $v'(x) = e^{-x} \implies v(x) = -e^{-x}$.
$I_1 = [-xe^{-x}]_0^1 - \int_0^1 -e^{-x} dx$
$I_1 = (-e^{-1} - 0) + [-e^{-x}]_0^1$
$I_1 = -\frac{1}{e} + (-e^{-1} - (-1)) = -\frac{2}{e} + 1$.

\textbf{2. Encadrement et limite}
Sur $[0, 1]$, $x \geq 0$ et $e^{-x} > 0$, donc $x^n e^{-x} \geq 0$. D'où $I_n \geq 0$.
De plus, la fonction $x \mapsto e^{-x}$ est décroissante sur $[0, 1]$, donc $e^{-x} \leq e^0 = 1$.
Donc $x^n e^{-x} \leq x^n$.
En intégrant : $\int_0^1 x^n e^{-x} dx \leq \int_0^1 x^n dx$.
$I_n \leq [\frac{x^{n+1}}{n+1}]_0^1 = \frac{1}{n+1}$.
Conclusion : $0 \leq I_n \leq \frac{1}{n+1}$.
Théorème des gendarmes : $\lim_{n \to +\infty} I_n = 0$.

\textbf{3. Relation de récurrence}
$I_{n+1} = \int_0^1 x^{n+1} e^{-x} dx$.
IPP : $u(x) = x^{n+1} \implies u'(x) = (n+1)x^n$.
$v'(x) = e^{-x} \implies v(x) = -e^{-x}$.
$I_{n+1} = [-x^{n+1}e^{-x}]_0^1 - \int_0^1 (n+1)x^n (-e^{-x}) dx$
$I_{n+1} = (-1^{n+1}e^{-1} - 0) + (n+1) \int_0^1 x^n e^{-x} dx$
$I_{n+1} = -\frac{1}{e} + (n+1)I_n$.
C.Q.F.D.

\textbf{4. Calcul d'aire}
$f(x) = x e^{-x}$. Sur $[0, 1]$, $f(x) \geq 0$.
L'aire est donc $\mathcal{A} = \int_0^1 f(x) dx = I_1$ (en unités d'aire).
$\mathcal{A} = 1 - \frac{2}{e}$ u.a.
\end{correction}
