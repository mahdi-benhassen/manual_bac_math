\chapter{Fonctions Transcendantes}

\section{Rappels Théoriques}

\subsection{Fonction Logarithme Népérien ($\ln$)}
\begin{definition}
La fonction logarithme népérien, notée $\ln$, est la primitive de la fonction $x \mapsto \frac{1}{x}$ sur $]0, +\infty[$ qui s'annule en 1.
\end{definition}

\textbf{Propriétés algébriques :}
Pour tout $a > 0$ et $b > 0$ :
\begin{itemize}
    \item $\ln(a \times b) = \ln(a) + \ln(b)$
    \item $\ln(\frac{a}{b}) = \ln(a) - \ln(b)$
    \item $\ln(\frac{1}{a}) = -\ln(a)$
    \item $\ln(a^n) = n \ln(a)$ (pour tout $n \in \mathbb{Q}$)
    \item $\ln(\sqrt{a}) = \frac{1}{2} \ln(a)$
\end{itemize}

\textbf{Limites usuelles :}
\begin{itemize}
    \item $\lim_{x \to +\infty} \ln(x) = +\infty$
    \item $\lim_{x \to 0^+} \ln(x) = -\infty$
    \item \textbf{Croissance comparée ($+\infty$) :} $\lim_{x \to +\infty} \frac{\ln x}{x} = 0$ (et plus généralement $\frac{\ln x}{x^n} \to 0$)
    \item \textbf{Croissance comparée ($0^+$) :} $\lim_{x \to 0^+} x \ln x = 0$
    \item \textbf{Taux d'accroissement en 1 :} $\lim_{x \to 1} \frac{\ln x}{x - 1} = 1$ (ou $\lim_{h \to 0} \frac{\ln(1+h)}{h} = 1$)
\end{itemize}

\textbf{Dérivée :}
Si $u$ est une fonction dérivable et strictement positive sur $I$, alors la fonction $\ln(u)$ est dérivable sur $I$ et :
\[ (\ln(u))' = \frac{u'}{u} \]

\subsection{Fonction Exponentielle ($\exp$)}
\begin{definition}
La fonction exponentielle, notée $\exp$ ou $e^x$, est la fonction réciproque de la fonction $\ln$. Elle est définie sur $\mathbb{R}$ et à valeurs dans $]0, +\infty[$.
\end{definition}

\textbf{Propriétés algébriques :}
Pour tout $a, b \in \mathbb{R}$ :
\begin{itemize}
    \item $e^{a+b} = e^a \times e^b$
    \item $e^{a-b} = \frac{e^a}{e^b}$
    \item $e^{-a} = \frac{1}{e^a}$
    \item $(e^a)^n = e^{na}$
\end{itemize}
Relation fondamentale : $y = e^x \iff x = \ln y$ (pour $y > 0$).

\textbf{Limites usuelles :}
\begin{itemize}
    \item $\lim_{x \to +\infty} e^x = +\infty$
    \item $\lim_{x \to -\infty} e^x = 0$
    \item \textbf{Croissance comparée ($+\infty$) :} $\lim_{x \to +\infty} \frac{e^x}{x} = +\infty$ (l'exponentielle l'emporte sur la puissance)
    \item \textbf{Croissance comparée ($-\infty$) :} $\lim_{x \to -\infty} x e^x = 0$
    \item \textbf{Taux d'accroissement en 0 :} $\lim_{x \to 0} \frac{e^x - 1}{x} = 1$
\end{itemize}

\textbf{Dérivée :}
Si $u$ est dérivable sur $I$, alors $e^u$ est dérivable sur $I$ et :
\[ (e^u)' = u' e^u \]
En particulier, $(e^x)' = e^x$.

\subsection{Fonctions Puissances}
Pour tout $\alpha \in \mathbb{R}$, on définit la fonction puissance sur $]0, +\infty[$ par :
\[ x^\alpha = e^{\alpha \ln x} \]
\textbf{Dérivée :} $(x^\alpha)' = \alpha x^{\alpha - 1}$.

\section{Exercices de Compréhension}

\begin{rappel}
\textbf{Objectif :} Manipuler les propriétés algébriques et lever des indéterminations.
\end{rappel}

\textbf{Exercice 3.1 : Simplification et Résolution}
\begin{enumerate}
    \item Simplifier $A = \ln(e^3) + \ln(\sqrt{e}) - e^{\ln 2}$.
    \item Résoudre dans $\mathbb{R}$ l'inéquation $e^{2x} - 3e^x + 2 \leq 0$.
\end{enumerate}

\begin{correction}
\begin{enumerate}
    \item $A = 3\ln(e) + \frac{1}{2}\ln(e) - 2$. Comme $\ln(e) = 1$, $A = 3 + 0.5 - 2 = 1.5$.
    \item Posons $X = e^x$. On a $X^2 - 3X + 2 \leq 0$.
    Racines de $X^2 - 3X + 2 = 0$ : $\Delta = 9 - 8 = 1$. $X_1 = 1$, $X_2 = 2$.
    Le polynôme est négatif entre les racines : $1 \leq X \leq 2$.
    Donc $1 \leq e^x \leq 2$.
    Par croissance de la fonction $\ln$ : $\ln(1) \leq \ln(e^x) \leq \ln(2)$.
    \[ 0 \leq x \leq \ln 2 \]
    $S = [0, \ln 2]$.
\end{enumerate}
\end{correction}

\textbf{Exercice 3.2 : Limites}
Calculer les limites suivantes :
\begin{enumerate}
    \item $\lim_{x \to +\infty} (x - \ln x)$
    \item $\lim_{x \to -\infty} (x + 1)e^x$
\end{enumerate}

\begin{correction}
\begin{enumerate}
    \item Forme indéterminée $\infty - \infty$.
    Factorisons par $x$ : $x - \ln x = x(1 - \frac{\ln x}{x})$.
    Or $\lim_{x \to +\infty} \frac{\ln x}{x} = 0$.
    Donc $\lim_{x \to +\infty} x(1 - 0) = +\infty$.
    \item $\lim_{x \to -\infty} (x + 1)e^x = \lim_{x \to -\infty} xe^x + e^x$.
    On sait que $\lim_{x \to -\infty} xe^x = 0$ et $\lim_{x \to -\infty} e^x = 0$.
    Donc la limite est $0$.
\end{enumerate}
\end{correction}

\textbf{Exercice 3.3 : Étude de fonction} \\
Soit $f$ définie sur $]0, +\infty[$ par $f(x) = \frac{\ln x}{x}$.
\begin{enumerate}
    \item Étudier les variations de $f$.
    \item Déduire que pour tout $n \geq 3$, $n^{n+1} > (n+1)^n$.
\end{enumerate}

\begin{correction}
\textbf{1. Variations}
$f$ est dérivable sur $]0, +\infty[$.
$f'(x) = \frac{\frac{1}{x} \cdot x - \ln x \cdot 1}{x^2} = \frac{1 - \ln x}{x^2}$.
Signe de $f'(x)$ :
$1 - \ln x > 0 \iff \ln x < 1 \iff x < e$.
$f$ est croissante sur $]0, e]$ et décroissante sur $[e, +\infty[$.
Maximum en $e$ : $f(e) = \frac{1}{e}$.

\textbf{2. Inégalité}
On veut comparer $n^{n+1}$ et $(n+1)^n$.
Passons au logarithme : $(n+1)\ln n$ vs $n \ln(n+1)$.
Divisons par $n(n+1)$ (positif) : $\frac{\ln n}{n}$ vs $\frac{\ln(n+1)}{n+1}$.
C'est comparer $f(n)$ et $f(n+1)$.
Pour $n \geq 3$, on est dans l'intervalle $[3, +\infty[$ où $f$ est décroissante (car $3 > e \approx 2.718$).
Donc $n < n+1 \implies f(n) > f(n+1)$.
Donc $\frac{\ln n}{n} > \frac{\ln(n+1)}{n+1} \implies (n+1)\ln n > n \ln(n+1) \implies \ln(n^{n+1}) > \ln((n+1)^n)$.
D'où $n^{n+1} > (n+1)^n$.
\end{correction}

\textbf{Exercice 3.4 : Vrai ou Faux ?}
Répondre par Vrai ou Faux en justifiant.
\begin{enumerate}
    \item Pour tout $x \in \mathbb{R}$, $e^x > x$.
    \item L'équation $\ln(x) = -2$ n'admet pas de solution réelle.
\end{enumerate}

\begin{correction}
\begin{enumerate}
    \item \textbf{Vrai.} Étudier $h(x) = e^x - x$. $h'(x) = e^x - 1$. Minimum en 0 : $h(0) = 1$. Donc $h(x) \geq 1 > 0$.
    \item \textbf{Faux.} La fonction $\ln$ est une bijection de $]0, +\infty[$ sur $\mathbb{R}$. $-2 \in \mathbb{R}$, donc il existe une solution unique $x = e^{-2} = 1/e^2$.
\end{enumerate}
\end{correction}

\autoeval{
Je connais les propriétés algébriques de $\ln$ et $\exp$ & & \\ \hline
Je sais résoudre des équations avec $\ln$ et $\exp$ & & \\ \hline
Je maîtrise les limites par croissance comparée & & \\ \hline
Je sais dériver $\ln(u)$ et $e^u$ & & \\
}

\section{Exercices Type Bac}

\begin{exobac}
\textbf{Sujet : Étude d'une fonction logarithme auxiliaire}

\textbf{Partie A}
Soit $g$ la fonction définie sur $]0, +\infty[$ par $g(x) = x^2 + 1 - \ln x$.
\begin{enumerate}
    \item Étudier les variations de $g$.
    \item En déduire le signe de $g(x)$ sur $]0, +\infty[$.
\end{enumerate}

\textbf{Partie B}
Soit $f$ la fonction définie sur $]0, +\infty[$ par $f(x) = x + \frac{\ln x}{x}$.
\begin{enumerate}
    \item Calculer les limites de $f$ en $0^+$ et en $+\infty$.
    \item Montrer que pour tout $x > 0$, $f'(x) = \frac{g(x)}{x^2}$.
    \item Dresser le tableau de variation de $f$.
    \item Montrer que la droite $\Delta$ d'équation $y = x$ est asymptote oblique à la courbe $\mathcal{C}_f$ au voisinage de $+\infty$.
    \item Étudier la position relative de $\mathcal{C}_f$ et $\Delta$.
\end{enumerate}
\end{exobac}

\begin{correction}
\textbf{Partie A}
\begin{enumerate}
    \item $g'(x) = 2x - \frac{1}{x} = \frac{2x^2 - 1}{x}$.
    Sur $]0, +\infty[$, le signe dépend de $2x^2 - 1$.
    $2x^2 - 1 = 0 \iff x^2 = 1/2 \iff x = \frac{1}{\sqrt{2}}$ (car $x>0$).
    $g$ est décroissante sur $]0, \frac{1}{\sqrt{2}}]$ et croissante sur $[\frac{1}{\sqrt{2}}, +\infty[$.
    \item Minimum de $g$ en $x_0 = \frac{1}{\sqrt{2}}$ :
    $g(x_0) = \frac{1}{2} + 1 - \ln(2^{-1/2}) = 1.5 + \frac{1}{2}\ln 2$.
    Comme $\ln 2 > 0$, le minimum est strictement positif.
    Donc \textbf{pour tout $x > 0$, $g(x) > 0$}.
\end{enumerate}

\textbf{Partie B}
\begin{enumerate}
    \item \textbf{En $0^+$ :} $\lim x = 0$ et $\lim \frac{\ln x}{x} = -\infty$ (car $\ln x \to -\infty$ et $1/x \to +\infty$). Donc $\lim_{0^+} f(x) = -\infty$. (Asymptote verticale $x=0$).
    \textbf{En $+\infty$ :} $\lim x = +\infty$ et $\lim \frac{\ln x}{x} = 0$. Donc $\lim_{+\infty} f(x) = +\infty$.
    
    \item $f(x) = x + \frac{\ln x}{x}$.
    $f'(x) = 1 + \frac{\frac{1}{x} \cdot x - \ln x \cdot 1}{x^2} = 1 + \frac{1 - \ln x}{x^2} = \frac{x^2 + 1 - \ln x}{x^2} = \frac{g(x)}{x^2}$.
    
    \item Comme $g(x) > 0$ et $x^2 > 0$, alors $f'(x) > 0$ sur $]0, +\infty[$.
    $f$ est \textbf{strictement croissante} sur $]0, +\infty[$.
    
    \item Calculons la limite de la différence :
    $f(x) - x = \frac{\ln x}{x}$.
    $\lim_{x \to +\infty} (f(x) - x) = 0$.
    Donc la droite $\Delta : y = x$ est bien \textbf{asymptote oblique}.
    
    \item Signe de $f(x) - x = \frac{\ln x}{x}$.
    Sur $]0, 1[$, $\ln x < 0 \implies \mathcal{C}_f$ est au-dessous de $\Delta$.
    Sur $]1, +\infty[$, $\ln x > 0 \implies \mathcal{C}_f$ est au-dessus de $\Delta$.
    Intersection au point $(1, 1)$.
\end{enumerate}
\end{correction}
