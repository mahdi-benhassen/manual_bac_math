\chapter{Dénombrement et Probabilités}

\section{Rappels Théoriques}

\subsection{Dénombrement}
Soit $E$ un ensemble fini à $n$ éléments.
\begin{itemize}
    \item \textbf{p-uplet (avec ordre et avec répétition) :} Tirage de $p$ éléments parmi $n$ avec remise.
    Nombre de possibilités : $n^p$.
    
    \item \textbf{Arrangement (avec ordre et sans répétition) :} Tirage de $p$ éléments parmi $n$ sans remise.
    Nombre de possibilités : $A_n^p = \frac{n!}{(n-p)!} = n(n-1)\dots(n-p+1)$.
    
    \item \textbf{Permutation :} Arrangement de $n$ éléments parmi $n$.
    Nombre de possibilités : $P_n = n!$.
    
    \item \textbf{Combinaison (sans ordre et sans répétition) :} Tirage simultané de $p$ éléments parmi $n$.
    Nombre de possibilités : $C_n^p = \binom{n}{p} = \frac{n!}{p!(n-p)!}$.
\end{itemize}

\subsection{Probabilités}
\begin{itemize}
    \item \textbf{Propriétés :} $P(\emptyset) = 0$, $P(\Omega) = 1$, $0 \leq P(A) \leq 1$.
    $P(A \cup B) = P(A) + P(B) - P(A \cap B)$.
    $P(\bar{A}) = 1 - P(A)$.
    
    \item \textbf{Probabilité Conditionnelle :}
    $P(A|B) = P_B(A) = \frac{P(A \cap B)}{P(B)}$ (si $P(B) \neq 0$).
    
    \item \textbf{Indépendance :}
    $A$ et $B$ indépendants $\iff P(A \cap B) = P(A) \times P(B)$.
    
    \item \textbf{Formule des Probabilités Totales :}
    Si $(B_1, B_2, \dots, B_n)$ est une partition de l'univers :
    $P(A) = \sum_{i=1}^n P(A \cap B_i) = \sum_{i=1}^n P(A|B_i) \times P(B_i)$.
\end{itemize}

\section{Exercices de Compréhension}

\begin{rappel}
\textbf{Objectif :} Choisir le bon outil de dénombrement et calculer une probabilité conditionnelle.
\end{rappel}

\textbf{Exercice 12.1 : Urne}
Une urne contient 5 boules rouges et 3 boules vertes.
1. On tire simultanément 3 boules. Combien de tirages possibles ?
2. On tire successivement 3 boules avec remise. Combien de tirages possibles ?

\begin{correction}
1. Tirage simultané $\implies$ Combinaison.
Total boules = 8. On tire 3.
$C_8^3 = \frac{8 \times 7 \times 6}{3 \times 2 \times 1} = 56$.

2. Tirage successif avec remise $\implies$ p-uplet.
$8^3 = 512$.
\end{correction}

\textbf{Exercice 12.2 : Probabilité conditionnelle}
Dans une population, 40\% sont des hommes ($H$) et 60\% des femmes ($F$).
10\% des hommes sont fumeurs ($Fu$) et 5\% des femmes sont fumeuses.
On choisit une personne au hasard. Quelle est la probabilité qu'elle soit fumeuse ?

\begin{correction}
Arbre pondéré :
- $P(H) = 0.4$, $P(F) = 0.6$.
- $P(Fu|H) = 0.1$, $P(Fu|F) = 0.05$.
Formule des probabilités totales :
$P(Fu) = P(Fu \cap H) + P(Fu \cap F)$
$P(Fu) = P(H) \times P(Fu|H) + P(F) \times P(Fu|F)$
$P(Fu) = 0.4 \times 0.1 + 0.6 \times 0.05$
$P(Fu) = 0.04 + 0.03 = 0.07$.
Soit 7\%.
\end{correction}

\section{Exercices Type Bac}

\begin{exobac}
\textbf{Sujet : Probabilités et Suite}

Une urne contient 2 boules blanches et 1 boule noire.
On effectue une suite de tirages d'une boule avec remise.
On s'arrête dès que l'on obtient une boule noire.
Soit $n$ un entier non nul. On note $E_n$ l'événement "on s'arrête au $n$-ième tirage".

1. Calculer la probabilité de tirer une boule noire ($p$) et une boule blanche ($q$).
2. Calculer $P(E_1)$, $P(E_2)$ et $P(E_3)$.
3. Exprimer $P(E_n)$ en fonction de $n$.
4. Calculer la probabilité $S_n$ de s'arrêter au plus tard au $n$-ième tirage.
5. Déterminer la limite de $S_n$. Interpréter.
\end{exobac}

\begin{correction}
\textbf{1. Probabilités élémentaires}
Total = 3 boules. 1 Noire, 2 Blanches.
$p = P(N) = 1/3$.
$q = P(B) = 2/3$.

\textbf{2. Premiers événements}
- $E_1$ : "N au 1er tirage". $P(E_1) = 1/3$.
- $E_2$ : "B au 1er, N au 2ème". $P(E_2) = (2/3) \times (1/3) = 2/9$.
- $E_3$ : "B, B, N". $P(E_3) = (2/3)^2 \times (1/3) = 4/27$.

\textbf{3. Formule générale}
Pour s'arrêter au rang $n$, il faut avoir tiré $n-1$ boules blanches, puis une noire.
$P(E_n) = q^{n-1} \times p = (\frac{2}{3})^{n-1} \times \frac{1}{3}$.

\textbf{4. Somme $S_n$}
$S_n = P(E_1 \cup E_2 \cup \dots \cup E_n) = \sum_{k=1}^n P(E_k)$.
$S_n = \sum_{k=1}^n \frac{1}{3} (\frac{2}{3})^{k-1}$.
C'est la somme des termes d'une suite géométrique de premier terme $1/3$ et de raison $2/3$.
$S_n = \frac{1}{3} \frac{1 - (2/3)^n}{1 - 2/3} = \frac{1}{3} \frac{1 - (2/3)^n}{1/3} = 1 - (\frac{2}{3})^n$.

\textbf{5. Limite}
Comme $-1 < 2/3 < 1$, $\lim (2/3)^n = 0$.
Donc $\lim S_n = 1$.
Cela signifie que l'on est "presque sûr" de finir par tirer une boule noire si on joue indéfiniment.
\end{correction}

\textbf{Exercice 12.3 : Dénombrement et Anagrammes}
Combien de mots (ayant un sens ou non) peut-on former avec les lettres du mot :
1. MATHS
2. ANANAS

\begin{correction}
\textbf{1. MATHS}
5 lettres distinctes.
Il s'agit de permutations de 5 éléments.
Nombre = $5! = 120$.

\textbf{2. ANANAS}
6 lettres au total : 3 A, 2 N, 1 S.
C'est une permutation avec répétition.
Nombre = $\frac{6!}{3! \times 2! \times 1!} = \frac{720}{6 \times 2 \times 1} = \frac{720}{12} = 60$.
\end{correction}

\textbf{Exercice 12.4 : Vrai ou Faux ?}
Répondre par Vrai ou Faux en justifiant.
\begin{enumerate}
    \item Si $A$ et $B$ sont indépendants, alors $P(A \cup B) = P(A) + P(B)$.
    \item Le nombre de tirages simultanés de 3 boules parmi 10 est $10^3$.
\end{enumerate}

\begin{correction}
\begin{enumerate}
    \item \textbf{Faux.} $P(A \cup B) = P(A) + P(B) - P(A \cap B)$. Si indépendants, $P(A \cap B) = P(A)P(B) \neq 0$ (sauf cas triviaux). La formule proposée est pour des événements incompatibles.
    \item \textbf{Faux.} Simultané = Combinaison $C_{10}^3$. $10^3$ c'est pour des tirages successifs avec remise.
\end{enumerate}
\end{correction}

\autoeval{
Je sais distinguer combinaison, arrangement et p-uplet & & \\ \hline
Je sais calculer une probabilité avec un arbre pondéré & & \\ \hline
Je connais la formule des probabilités totales & & \\ \hline
Je sais calculer une somme de probabilités (suite géométrique) & & \\
}

\section{Exercices Type Bac}
